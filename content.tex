\documentclass[]{scrartcl}
\usepackage{lmodern}
\usepackage{amssymb,amsmath,amsopn}
\DeclareMathOperator*{\argmin}{arg\,min}
\usepackage{ifxetex,ifluatex}
\usepackage{fixltx2e} % provides \textsubscript
\ifnum 0\ifxetex 1\fi\ifluatex 1\fi=0 % if pdftex
  \usepackage[T1]{fontenc}
  \usepackage[utf8]{inputenc}
\else % if luatex or xelatex
  \ifxetex
    \usepackage{mathspec}
  \else
    \usepackage{fontspec}
  \fi
  \defaultfontfeatures{Ligatures=TeX,Scale=MatchLowercase}
\fi
% use upquote if available, for straight quotes in verbatim environments
\IfFileExists{upquote.sty}{\usepackage{upquote}}{}
% use microtype if available
\IfFileExists{microtype.sty}{%
\usepackage{microtype}
\UseMicrotypeSet[protrusion]{basicmath} % disable protrusion for tt fonts
}{}
\usepackage{hyperref}
\PassOptionsToPackage{usenames,dvipsnames}{color} % color is loaded by hyperref
\hypersetup{unicode=true,
            colorlinks=true,
            linkcolor=Maroon,
            citecolor=Blue,
            urlcolor=Blue,
            breaklinks=true}
\urlstyle{same}  % don't use monospace font for urls
\IfFileExists{parskip.sty}{%
\usepackage{parskip}
}{% else
\setlength{\parindent}{0pt}
\setlength{\parskip}{6pt plus 2pt minus 1pt}
}
\setlength{\emergencystretch}{3em}  % prevent overfull lines
\providecommand{\tightlist}{%
  \setlength{\itemsep}{0pt}\setlength{\parskip}{0pt}}
\setcounter{secnumdepth}{0}
% Redefines (sub)paragraphs to behave more like sections
\ifx\paragraph\undefined\else
\let\oldparagraph\paragraph
\renewcommand{\paragraph}[1]{\oldparagraph{#1}\mbox{}}
\fi
\ifx\subparagraph\undefined\else
\let\oldsubparagraph\subparagraph
\renewcommand{\subparagraph}[1]{\oldsubparagraph{#1}\mbox{}}
\fi

\title{Reading the Deep Learning book}
\providecommand{\subtitle}[1]{}
\subtitle{Chapter 2}
\author{Andreas Kirsch}
\date{4/4/2017}

\begin{document}
\maketitle

{
\hypersetup{linkcolor=black}
\setcounter{tocdepth}{3}
\tableofcontents
}
This are my notes and observations from reading the
\href{http://www.deeplearningbook.org/contents/linear_algebra.html}{Linear
Algebra chapter} of the \href{http://www.deeplearningbook.org/}{Deep
Learning} book.

The following notes are presented in order of value to the reader. I
start with a discussion of the Moore-Penrose pseudoinverse, followed by
a short reflection on ``broadcasting''.

\subsection{Moore-Penrose
pseudoinverse}\label{moore-penrose-pseudoinverse}

The Moore-Penrose pseudoinverse is defined as:

\begin{quote}
\href{http://www.deeplearningbook.org/contents/linear_algebra.html\#pff}{\[ A^{+}:=\lim _{\alpha \searrow  0}\left( A^{T}A+\alpha I\right) ^{-1}A^{T} \]}
\end{quote}

You can click on the formula to jump into the book. (It's really awesome
that all of it is online and you can deep link to pages.)

This formula is mystifying. What is its origin? And why does it do what
the book says it does?

We can get a feeling for its origin by looking at the equation \(Ax=b\)
that it solves:

\[
\begin{align*} Ax &=b &|& A^{T}\cdot \\ A^{T}Ax &=A^{T}b &|& \left( A^{T}A\right) ^{-1} \cdot \\ \left( A^{T}A\right) ^{-1}A^{T}Ax &=\left( A^{T}A\right) ^{-1}A^{T}b & & \\ x &=\left( A^{T}A\right) ^{-1}A^{T}b  & & \end{align*}
\]

So we start with the equation that we want to solve and find a more
complex form for solving for \(x\) than the usual left-multiplication
with the inverse \(A^{-1}\). This is already useful if \(A^T A\) is
invertible and \(A\) is not square (because matrix inverses only exist
for square matrices).

This already looks similar to the pseudo-inverse, except for the
\(+\alpha I\) term.

\subsubsection{Aside: there is also a right
pseudo-inverse}\label{aside-there-is-also-a-right-pseudo-inverse}

We can guess its form by starting with:

\[
\begin{align*} yA&=b&|&\cdot A^{T}\\ yAA^{T}&=bA^{T}&|&\cdot \left( AA^{T}\right) ^{-1}\\ yAA^{T}\left( AA^{T}\right) ^{-1}&=bA^{T}\left( AA^{T}\right) ^{-1}\\ y&=bA^{T}\left( AA^{T}\right) ^{-1}\end{align*}
\]

From this, we can make an educated guess:

\[
^{+}A = A^{T}\lim _{\alpha \searrow 0}\left( AA^{T}+\alpha I\right) ^{-1}
\]

Since we cannot always invert \(A^TA\) (respectively \(AA^T\)), an
obvious question is:

\subsubsection{\texorpdfstring{Why can we invert \(AA^T+ \alpha I\) for
positive
\(\alpha\)?}{Why can we invert AA\^{}T+ \textbackslash{}alpha I for positive \textbackslash{}alpha?}}\label{why-can-we-invert-aat-alpha-i-for-positive-alpha}

Let's examine this.
\href{https://en.wikipedia.org/wiki/Invertible_matrix}{For a matrix to
be invertible}, its kernel has to only contain the zero vector:

\[\ker \left( A^{T}A+\alpha I\right) =\left\{ 0\right\}\]

To prove that this is the case for \(\left( AA^{T}+\alpha I\right)\), we
need to show that:

\[  \left( A^{T}A+\alpha I\right) v = 0 \implies v = 0\]

So starting with the left side, we can rephrase it as follows:

\[ \begin{align*} 
& & \left( A^{T}A+\alpha I\right) v&= 0\\ 
& \Leftrightarrow &A^{T}Av+\alpha Iv&= 0\\ 
& \Leftrightarrow &A^{T}Av &= -\alpha v
\end{align*} \]

If \(v \ne 0\), this would mean that \(-\alpha\) is a negative
\href{https://en.wikipedia.org/wiki/Eigenvalues_and_eigenvectors}{eigenvalue}
of \(A^TA\) as \(\alpha\) is \(>0\).

\paragraph{\texorpdfstring{Can \(A^TA\) have negative
eigenvalues?}{Can A\^{}TA have negative eigenvalues?}}\label{can-ata-have-negative-eigenvalues}

Let's assume \(v \ne 0\) and \(-\alpha\) is a negative eigenvalue, that
is \(A^{T}Av = -\alpha v\) holds (and \(\alpha > 0\)). We can
left-multiply with \(v^T\) and obtain:

\[ 
\begin{align*} & \Leftrightarrow v^{T}A^{T}Av=v^{T}\left( -\alpha v\right) \\ & \Leftrightarrow \left\| Av\right\| _{2}^{2}=-\alpha v^{T}v=-\alpha \left\| v\right\| _{2}^{2}\\ & \Leftrightarrow -\alpha =\dfrac {\left\| Av\right\| _{2}^{2}} {\left\| v\right\| _{2}^{2}}\geq 0
\end{align*} 
\]

(We can divide by \(\left\| v\right\| _{2}^{2}\) because we assume
\(v\ne0\).) Now this means, that \(-\alpha\) has to be \(>0\), so it is
not a negative eigenvalue and a contradiction to our initial assumption
\(\alpha > 0\). In fact, we have just shown that \(A^TA\) in general can
only have non-negative eigenvalues. (We can show the same for \(AA^T\)
the same way.)

Because \(A^TA\) cannot have negative eigenvalues, this means that \(v\)
cannot be \(\ne 0\), so \(v = 0\). This is what we wanted to show, and
this means that the kernel only contains the zero vector, and
\(AA^T + \alpha I\) is invertible for \(\alpha > 0\).

\subsubsection{\texorpdfstring{What if \(A^T A\) was
invertible?}{What if A\^{}T A was invertible?}}\label{what-if-at-a-was-invertible}

For a moment, let's consider the case when \(A^T A\) is invertible: Is
the Moore-Penrose pseudoinverse then equal to
\(\left (A^T A \right)^{-1} A^T\)?

Matrix inversion is continuous in the space of invertible matrices. You
might remember the definition of continuity from school.
\href{https://en.wikipedia.org/wiki/Continuous_function\#Definition_in_terms_of_limits_of_sequences}{A
better definition of continuity} is that it means being able to swap
function and limit application:

\begin{quote}
A function \(f\) is continuous on its domain iff for any convergent
series \(x_n \to x\) for which \(f \left( x \right )\) is defined
\[ \lim_{n \to \infty } f \left ( x_n \right ) = f \left ( \lim_{n \to \infty} x_n \right ) = f \left( x \right).\]
\end{quote}

Assuming \(A^T A\) is invertible and using the fact that matrix
inversion is continuous for invertible matrices, we now see:

\[ \begin{align*} 
\lim _{\alpha \searrow 0}\left( A^{T} A +\alpha I\right) ^{-1} A^T &= \left( \lim _{\alpha \searrow 0} A^{T} A +\alpha I\right) ^{-1} A^T \\
&= \left( A^{T}A+0 \, I\right) ^{-1} A^T \\
&= \left( A^{T}A\right) ^{-1} A^T
\end{align*}  \]

So in this special case, the pseudoinverse is exactly the solution we
have come up with ourselves.

\subsubsection{Two properties of the
pseudoinverse}\label{two-properties-of-the-pseudoinverse}

There are two properties mentioned in text that are interesting but not
obvious:

\begin{quote}
\href{http://www.deeplearningbook.org/contents/linear_algebra.html\#pf10}{When
\(A\) has more columns than rows, then solving a linear equation using
the pseudoinverse provides one of the many possible solutions.
Specifically, it provides the solution \(x=A^+y\) with minimal Euclidean
norm \(\left\| x \right\|_2\) among all possible solutions. When \(A\)
has more rows than columns, it is possible for there to be no solution.
In this case, using the pseudoinverse gives us the \(x\) for which
\(Ax\) is as close as possible to \(y\) in terms of Euclidean norm
\(\left\| Ax - y \right\|_2\).}
\end{quote}

These properties are not obvious and their deduction is enlightening
towards the chosen definition of the pseudoinverse, specifically the use
of the limit and the constraint of \(\alpha\) to be \(>0\).

\paragraph{A related optimization
problem}\label{a-related-optimization-problem}

To prove the properties, let's look at the following regularized
minimum-least-squares problem. This is something that is formulated in
more detail in
\href{http://www.deeplearningbook.org/contents/numerical.html\#pf11}{Chapter
4 of the book}, but it is quite useful here:

\[\min _{x}\left| \left| Ax-b\right| \right| _{2}^{2}+\alpha \left\| x\right\| _{2}^{2}, \, \alpha > 0\]

Let's solve this optimization problem. First, we set up a cost function:

\[ \begin{align*} 
c_\alpha\left( x\right) &=\left\| Ax-b\right\| _{2}^{2}+\alpha \left\| x\right\| _{2}^{2}\\
& =\left( Ax-b\right) ^{T}\left( Ax-b\right) +\alpha x^{T}x\\
& =x^{T}AAx-2b^{T}Ax+b^{T}b+\alpha x^{T} x \end{align*} \\
\]

The first derivative of \(c_\alpha\) is:

\[ \nabla c_\alpha\left( x\right) =2A^{T}Ax-2A^{T}b+2\alpha x\]

And the second derivative is:

\[ H_\alpha \left ( x \right ) = 2A^T A + 2\alpha I\]

\(H_\alpha\) is
\href{https://en.wikipedia.org/wiki/Positive-definite_matrix}{positive
definite}, that is \(v^T H_\alpha \left ( x \right ) v \ge 0\) for all
\(v\). Why? We have already seen that \(A^T A\) only has non-negative
eigenvalues, so it is positive semidefinite by definition and
\(\alpha I\) is trivially positive definite for \(\alpha > 0\). The sum
of the two is positive definite again. Thus \(c\) is a
\href{https://en.wikipedia.org/wiki/Convex_function}{strictly convex
function}. This is along to lines of the one-dimensional case: when the
second derivative is \(> 0\) everywhere, the function is strictly
convex. Convex functions have a global minimum and, for strictly convex
functions, this global minimum is unique. So we know there is only
exactly one point that minimizes the cost function \(c_\alpha\).

We can determine this global minimum \(x^*_\alpha\) by solving
\(\nabla c_\alpha \left ( x \right ) = 0\):

\[ \begin{align*} 
& & \nabla c_\alpha\left( x\right) &= 0 \\
& \Leftrightarrow & A^{T}Ax-A^{T}b+ax&=0\\
& \Leftrightarrow &\left( A^{T}A+\alpha I\right) x&=A^{T}b\\
& \Leftrightarrow & x&=\left( A^{T}A+\alpha I\right) ^{-1}A^{T}b\end{align*}
\]

This is exactly the definition of the pseudoinverse without the limit.
So we have found:

\[
x^*_{\alpha} = \argmin _{x}\left| \left| Ax-b\right| \right| _{2}^{2}+\alpha \left\| x\right\| _{2}^{2} = \left( A^{T}A+\alpha I\right) ^{-1}A^{T}b
\]

with
\(c_\alpha \left (x^*_\alpha \right ) \leq c_\alpha \left ( x \right )\)
for all \(x\), and \(x^*_\alpha\) denotes the minimum point.

This expression is continuous in \(\alpha\), so we can take the limit
\(\alpha \searrow 0\). Remember, we need the constraint of
\(\alpha > 0\) for \(A^TA + \alpha I\) to be invertible, so we can only
take the limit from above. Taking the limit, we obtain:

\[
x^*= \lim_{x \searrow 0} \left( A^{T}A+\alpha I\right) ^{-1}A^{T}b
\]

with \(c \left (x^* \right ) \leq c \left ( x \right )\) for all \(x\)
with

\[c \left ( x \right ) = \lim_{\alpha \searrow 0} \left\| Ax-b\right\| _{2}^{2}+\alpha \left\| x\right\| _{2}^{2} = \left\| Ax-b\right\| _{2}^{2}.\]

What do we gain from this? Well for one, we now know that this
expression minimizes \(\left| \left| Ax-b\right| \right| _{2}^{2}\).
Furthermore, if there is a solution for \(A x =b\), we can easily use a
similar approach to see that the solution \(x^*\) is smaller under
Euclidean norm than any (other) solution \(\hat x\) for \(Ax=b\).

We do this in two steps. First, we observe that

\[A\hat x = b \Leftrightarrow A \hat x - b = 0 \Leftrightarrow \left\| A\widehat {x}-b\right\| _{2}^{2} = 0 \]

and, because \(x^*_\alpha\) minimizes \(c_\alpha\),

\[
\begin{align*}
& & c_{\alpha}\left( x_{\alpha }^{*}\right) &\leq c_{\alpha }\left( \hat {x}\right) \\
& \Leftrightarrow & c_{\alpha}\left( x_{\alpha }^{*}\right) &\leq \left\| A\hat {x}-b\right\| _{2}^{2}+\left\| \hat {x}\right\| _{2}^{2}=0+\alpha \left\| \hat x\right\| _{2}^{2} \\
& \Leftrightarrow & c_{\alpha}\left( x_{\alpha }^{*}\right) & \leq \alpha \left\| \hat x\right\| _{2}^{2} \end{align*}
\]

This expression is again continuous, so we can take the limit
\(\alpha \searrow 0\), and we see:

\[\begin{align*}
& c \left( x^{*} \right) \leq 0 \\
\Leftrightarrow &  \left\| Ax^{*}-b\right\| _{2}^{2} \le 0
\end{align*}
\]

Because norms are always non-negative, we have
\(0 \le \left\| Ax^{*}-b\right\| _{2}^{2} \le 0\), so
\(\left\| Ax^{*}-b\right\| _{2}^{2} = 0\). And we have observed above
that this is equivalent to \(Ax^{*} = b\). So if there is at least one
exact solution to the problem, we are sure to obtain an exact one, too.
To be fair, we could have deduced this in the previous section. However,
we can take another limit on the inequality
\(c_{\alpha}\left( x_{\alpha }^{*}\right) \leq \alpha \left\| \hat x\right\| _{2}^{2}\)
and obtain a more interesting result. This time, we only take the limit
\(\alpha \searrow 0\) of \(x^{*}_\alpha\), but keep \(c_\alpha\) fixed:

\[
\begin{align*}
&&c_{\alpha}\left( x_{\alpha }^{*}\right) &\leq \alpha \left\| \hat x\right\| _{2}^{2} \\
\Rightarrow  && c_{\alpha}\left( \lim_{\alpha \searrow 0} x_{\alpha }^{*}\right) &\leq \alpha \left\| \hat x\right\| _{2}^{2}  \\
\Leftrightarrow && c_{\alpha}\left( x^{*} \right) &\leq \alpha \left\| \hat x\right\| _{2}^{2} \\
\Leftrightarrow && \left\| Ax^{*}-b\right\| _{2}^{2}+\alpha \left\| x^{*}\right\| _{2}^{2} &\leq \alpha \left\| \hat x\right\| _{2}^{2} \\
\Rightarrow && \alpha \left\| x^{*}\right\| _{2}^{2} &\leq \alpha \left\| \hat x\right\| _{2}^{2} \\
\Leftrightarrow && \left\| x^{*}\right\| _{2}^{2} &\leq \left\| \hat x\right\| _{2}^{2} 
\end{align*}
\]

Here, we use \(\left\| Ax^{*}-b\right\| _{2}^{2} \ge 0\) to be able to
drop the term and \(\alpha > 0\) to preserve the direction of the
inequality, and we obtain the second property about the length of
\(\left \| x^{*} \right \|\).

Now we have proved both properties that were mentioned in the book. On
the one hand, if there are solutions, we will obtain one with minimal
norm. On the other hand, if there are no solutions, using the
pseudoinverse will provide us with an \(x\) that at least minimizes
\(\left| \left| Ax-b\right| \right| _{2}^{2}\).

To convince yourself that the expressions above are indeed continuous,
we can use the technical argument that we can rewrite
\(c_\alpha \left (x \right )\) into \(c \left (\alpha, x \right )\) and
see that it is a continuous function in two variables instead of going
the route of using functional analysis and treating \(c_\alpha\) as a
convergent series of functions.

\subsection{Broadcasting notation: oh
my\ldots{}}\label{broadcasting-notation-oh-my}

The broadcasting notation in the Deep Learning book is weird. For one
moment, let's ignore its origins from
\href{https://docs.scipy.org/doc/numpy-1.10.1/user/basics.broadcasting.html}{numpy},
and let's look at what's happening.

\begin{quote}
\href{http://www.deeplearningbook.org/contents/linear_algebra.html\#pf4}{In
the context of deep learning, we also use some less conventional
notation. We allow the addition of matrix and a vector, yielding another
matrix: \(C=A+b\), where \(C_{i,j}=A_{i,j}+b_j\). In other words, the
vector \(b\) is added to each row of the matrix. This shorthand
eliminates the need to define a matrix with \(b\) copied into each row
before doing the addition. This implicit copying of \(b\) to many
locations is called broadcasting.}
\end{quote}

Let's say \(A \in \mathbb{R}^{3 \times 3}\) and \(b \in \mathbb{R}^3\),
for example:

\[ A = \left[\begin{matrix}0 & 0 & 0\\1 & 1 & 1\\2 & 2 & 2\end{matrix}\right] , b = \left[\begin{matrix}1\\2\\3\end{matrix}\right] \]

Then, with broadcasting:

\[ A + b = \left[\begin{matrix}1 & 2 & 3\\2 & 3 & 4\\3 & 4 & 5\end{matrix}\right] \]

How do we get there? Essentially, we take the vector
\(b \in \mathbb{R}^3\), interpret it as a row vector, and add it to
every row of the matrix, so:

\[ A + b= A + \left[\begin{matrix}1\\1\\1\end{matrix}\right] b^T \]

Wouldn't it make more sense to write this as \(A + b^T\)? The reason for
the unintuitive notation lies in the details of broadcasting: whereas in
maths, a vector \(\mathbb{R}^3\) is identified as a column matrix
\(\mathbb{R}^{3 \times 1}\) , in the context of broadcasting it is
treated as a row matrix \(\mathbb{R}^{1 \times 3}\). This row matrix is
then repeated across the \(1\) dimension to make it into a
\(3 \times 3\) matrix that can be applied to \(A\). However, for matrix
multiplications, \(b\) continues to be treated as column matrix. This is
really confusing. So what's the reason for this ambiguity?

My best guess is that in ML contexts, data entries (examples) are often
treated as row vectors. For example, a design matrix stores a number of
examples, each in a separate row. So if you want to change the bias of
your examples, you want to add a bias vector as row-vector to each row.
Indeed, in chapter 8, broadcasting is used to express batch
normalization in eq (8.35):

\href{http://www.deeplearningbook.org/contents/optimization.html\#pf2d}{\[H'=\dfrac{H-\mu}{\sigma}\]}

I don't think this excuses the lack of clarity introduced with this
notation, but then again it is too late to change and make everybody use
a transposed design matrix\ldots{} :)

\subsection{Looking back and ahead}\label{looking-back-and-ahead}

There is much more I could write about, but I think these were the most
interesting bits from my notes. We have revisited convex functions,
continuity and limits to motivate the curious definition of the
Moore-Penrose pesudoinverse. Last but not least, we have looked at the
origin of the broadcasting notation and why it might be confusing to
someone who is new to ML. Next up: reading Chapter 3 and thinking about
probability theory.

Stay tuned,\\
Andreas

PS: Thanks to my friend Armin Krupp for suggestions and corrections of
the draft. All current mistakes and factual errors have been added
later.

PSS: This a gist repost of a post on my blog
\href{http://blog.blackhc.net/2017/03/dlb-chapter2}{http://blog.blackhc.net/2017/03/dlb-chapter2}. I wish Medium
was supporting LaTeX formulas properly\ldots{}

\end{document}
